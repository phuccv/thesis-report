\chapter*{Kết luận và hướng phát triển\markboth{Kết luận và hướng phát triển}{}}
\addcontentsline{toc}{chapter}{Kết luận và hướng phát triển}

%=======================================================================
\section*{Kết luận}

Đồ án tốt nghiệp với đề tài \textsc{Nghiên cứu và áp dụng thuật toán tìm kiếm cục bộ dựa trên ràng buộc để giải bài toán đóng thùng} đã hoàn thành các mục tiêu yêu cầu:
\begin{itemize}
	\item Nghiên cứu thuật toán tìm kiếm cục bộ để giải các bài toán thỏa mãn ràng buộc.
	\item Tìm hiểu và sử dụng thành thạo các thành phần của thư viện JOpenCBLS.
	\item Mô hình hóa và giải bài toán đóng thùng ở mức 2 chiều.
	\item Xây dựng ứng dụng Web sử dụng công nghệ JAVA và các công nghệ phía browser khác.
	\item Cài đặt các thuật toán tìm kiếm khác ngoài các thuật toán cài đặt sẵn để chạy thử, so sánh hiệu năng.
	\item Mở rộng thư viện JOpenCBLS để sử dụng tốt hơn trong các ứng dụng Web.
\end{itemize}
%=======================================================================

%=======================================================================
\section*{Hướng phát triển}

Đồ án đã nghiên cứu thành công phương pháp giải bài toán đóng thùng với phương pháp tìm kiếm cục bộ dựa trên ràng buộc sử dụng thư viện JOpenCBLS của TS. Phạm Quang Dũng. Nhưng trong phạm vi đồ án tôi không phân tích sử dụng được hết mọi kỹ năng của thuật toán tìm kiếm cục bộ dựa trên ràng buộc cũng như không thể nghiên cứu rộng hơn về lớp các bài toán đóng thùng. Sau đây là những hướng mở rộng nghiên cứu của đồ án này:
\begin{itemize}
	\item Nghiên cứu thêm các Heuristic tìm kiếm mới hiệu quả hơn, góp phần cải thiện hiệu năng tìm kiếm trong các bài toán tìm kiếm cục bộ dựa trên ràng buộc.
	\item Phát triển đầy đủ bộ công cụ mô hình cho thư viện JOpenCBLS.
	\item Cải thiện hiệu quả của thuật toán đã cài đặt để có thể sử dụng tốt hơn cho các bộ dữ liệu lớn hơn.
	\item Mở rộng bài toán đóng thùng 2 chiều sang bài toán $n$ chiều và thỏa mãn thêm nhiều ràng buộc hơn:
	\begin{itemize}
		\item Ràng buộc thứ tự đặt các vật vào thùng.
		\item Ràng buộc yêu cầu xếp cạnh nhau các vật.
	\end{itemize}
	\item Phát triển bài toán đóng thùng với yêu cầu mới: lựa chọn số lượng các vật cho vào thùng để tối ưu một hay nhiều mục tiêu nào đó.
\end{itemize}
Tiếp tục nghiên cứu các kỹ thuật nhằm tăng hiệu quả của thuật toán, giải quyết tốt hơn các bộ dữ liệu lớn.
