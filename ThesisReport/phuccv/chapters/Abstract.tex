\chapter*{Tóm tắt nột dung}%
\addcontentsline{toc}{chapter}{\numberline{}Tóm tắt nội dung}%
Quản lý file là một chức năng của hầu hết các hệ điều hành hiện nay. Tuy nhiên
hiện nay có nhiều hệ thống file (filesystem) khác nhau, mỗi hệ thống file có
những cách đọc, ghi dữ liệu khác nhau. Có những hệ thống file dành cho các
thiết bị lưu trữ vật lý, nhưng cũng có những hệ thống file dành riêng cho việc
chia sẻ qua mạng hay là lưu trữ trên bộ nhớ RAM.

Phần đầu bài báo cáo này tìm hiệu về hệ thống file ảo (VFS) trên Linux, một thành phần
của nhân Linux đóng vai trò là giao diện thống nhất tất cả các hệ thống file,
làm cho việc quản lý file trên Linux trở nên đơn giản hơn.

Phần hai của bài báo cáo tìm hiểu về bộ nhớ flash và hệ thống file F2FS dành
cho bộ nhớ flash. Cũng như nhiều hệ thống file khác trên Linux, cài đặt của
F2FS cũng sử dụng giao diện cung cấp bởi VFS. Qua việc tìm hiểu F2FS, ta sẽ
thấy quản lý file không phải chỉ cần quan tâm đến mức phần mềm mà còn phải biết tận dụng
các ưu điểm và khắc phục các nhược điểm của thiết bị lưu trữ phần cứng.
