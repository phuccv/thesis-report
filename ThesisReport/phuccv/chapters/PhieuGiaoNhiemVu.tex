\addcontentsline{toc}{chapter}{PHIẾU GIAO NHIỆM VỤ ĐỒ ÁN TỐT NGHIỆP}
\newgeometry{left=3.5cm,bottom=2cm,right=2cm,top=2cm,includefoot}
{\fontsize{14}{1}\selectfont\centering\bfseries
    PHIẾU GIAO NHIỆM VỤ ĐỒ ÁN TỐT NGHIỆP
\par}%

\fontsize{12}{14.4}
\selectfont

{\noindent
\setlength{\parindent}{0pt}

1. Thông tin về sinh viên\\
Họ và tên: Cao Văn Phúc \hspace{4.6cm} Email: phuccaotp@gmail.com\\
Lớp: CNTT4 \hspace{7.6cm} Hệ đào tạo: Chính quy\\
Đồ án tốt nghiệp được thực hiện tại: Phòng 902 - B1\\
Thời gian làm ĐATN: Từ ngày 20/02/2015 đến 20/08/2015\\
\\
2. Mục đích nội dung của ĐATN\\
Tìm hiểu giải thuật tìm kiếm cục bộ dựa trên ràng buộc và cài đặt
bài toán đóng thùng 2D (Bin Packing 2D) sử dụng thư viện JOpenCBLS.\\
\\
3. Các nhiệm vụ cụ thể của ĐATN
\begin{itemize}
    \item Tìm hiểu giải thuật tìm kiếm cục bộ giải các bài toán có ràng buộc,
    \item Tìm hiểu thư viện JOpenCBLS,
    \item Phát triển giải thuật tìm kiếm cục bộ để giải bài toán đóng thùng 2D,
    \item Cài đặt ứng dụng tương tác với người dùng,
    \item Thử nghiệm và đánh giá thư viện JOpenCBLS trên bài toán BP2D.
\end{itemize}

4. Lời cam đoan của sinh viên:\\
Tôi - \textit{Cao Văn Phúc} - cam kết ĐATN là công trình nghiên cứu của bản
thân tôi dưới sự hướng dẫn của học \textit{TS. Phạm Quang Dũng}.
Các kết quả nêu trong ĐATN là trung thực, không phải là sao chép toàn văn của
bất kỳ công trình nào khác.

\begin{table}[H]
    \centering
    \begin{tabular}{P{0.45\textwidth} P{0.45\textwidth}}
        & \textit{Hà Nội, \luaexec{tex.print(os.date("ngày \%d tháng \%m năm \%Y"))}}\newline
            Tác giả ĐATN\newline
            \newline
            \newline
          \textit{Cao Văn Phúc}\\
    \end{tabular}
\end{table}


5. Xác nhận của giáo viên hướng dẫn về mức độ hoàn thành của ĐATN và cho phép
bảo vệ:\\
\textit{Sinh viên đã hoàn thành tốt các nhiệm vụ của ĐATN đại học. Tôi đề nghị
        cho sinh viên được bảo vệ tốt nghiệp.}

\begin{table}[H]
    \centering
    \begin{tabular}{P{0.45\textwidth} P{0.45\textwidth}}
        %& \textit{Hà Nội, \luaexec{tex.print(os.date("ngày \%d tháng \%m năm \%Y"))}}\newline
        & \textit{Hà Nội, \luaexec{tex.print(os.date("ngày 28 tháng \%m năm \%Y"))}}\newline
            Giáo viên hướng dẫn\newline
            \newline
            \newline
          \textit{TS. Phạm Quang Dũng}\\
    \end{tabular}
\end{table}

}% noindent
\restoregeometry
